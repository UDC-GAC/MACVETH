% File              : manual.tex
% Author            : Marcos Horro <marcos.horro@udc.gal>
% Date              : Mar 24 Dec 2019 19:31:10 MST
% Last Modified Date: Mér 25 Dec 2019 18:33:20 MST
% Last Modified By  : Marcos Horro <marcos.horro@udc.gal>
\documentclass[a4paper,12pt]{memoir}
% For TeXstudio users:
% !TeX spellcheck = en_US

% Title Page
\title{\textbf{MACVETH}: \textbf{M}ulti-dimensional \textbf{A}rray
    \textbf{C}-compiler for \textbf{VE}ctorization and
    \textbf{T}ransformation in \textbf{H}PC applications}
\author{Marcos Horro}
\date{}

%%%%%%%%%%%%%%%%%%%%%%%%%%%%%%%%%%%%%
\begin{document}

\maketitle

%%%%%%%%%%%%%%%%%%%%%%%%%%%%%%%%%%%%%
\begin{abstract}
    This document presents the high-level picture of the MACVETH tool. Details regarding implementation and design decision are explained in detail.
\end{abstract}

\chapter*{Description}
MACVETH is a source-to-source compiler which targets the vectorization of C/C++
codes. This means that the input of this compiler is a non vectorized C/C++ code
and the output will be a not compiled SIMD-fashion C/C++ code.

\chapter*{MACVETH IR}
This compiler handles different abstraction levels or intermediate
representations for each step. As MACVETH is based in Clang~\cite{bib:clang},
the first abstraction layer lays on the Clang AST. This representation allows to
identify the regions of interest for the translation.

% Picture needed
Clang AST -> Three-Address Code -> CDAG -> VectorIR -> SIMD C/C++ instructions

\section{MACVETH Expressions: MVExpr}
Clang expressions are a monster, in a good way. They provide many possibilities
when it comes to parse exactly the code. Nonetheless, in our case we are
interested in a small set of operations, so we created basically a wrap for this
purpose. MVExpr is an abstract class that can be specialized for any type we
want to represent from the Clang AST. Besides, the idea of this class is to
provide a set of non-standard transformations for the expressions, e.g.
unrolling.

\section{Three-Address Code IR: TAC}
The Three-Address Code (TAC) representation used translates any statement into a
set of Single Stament Assignment using triples of the form $a = b \oplus c$,
where $\oplus$ can be any operation.

\section{Control Directered Acyclic Graph: CDAG}
When it comes to schedule the different TACs in the ROI of our program, we need
a representation which can handle the dependencies between the statements and
some structures that store the information about the placement of them in the
execution. For this purpose we use a CDAG. It basically represents the TACs as a
set of interconnected nodes, where each node can be a memory operation (load,
store) or any other type of operation (addition, multiplication, built-in
function, etc.).

\section{VectorIR}
In order to approach the different architectures when generating instructions,
we need a generic vector representation of the vector instructions we want to
have in our program. For this purpose, MACVETH uses the VectorIR, that basically
wraps a set of nodes from the CDAG onto a common structure which represent a vector operation.

To wrap up: VectorIR is a generic way of representing vector operations for the different
architectures. Because of this, at this stage there is no fusing operations or
any other kind of target-specific optimizations. The concrete backend will be in
charge of doing this.

\chapter*{SIMD backend: instruction generation}
For generating the SIMD instructions for each concrete architecture, MACVETH
uses the VectorIR described above.

\end{document}
%%%%%%%%%%%%%%%%%%%%%%%%%%%%%%%%%%%%%
