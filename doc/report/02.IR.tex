MACVETH compiler handles different abstraction levels or intermediate
representations in order to facilitate the vectorization process. As MACVETH is 
based in Clang/LLVM~\cite{bib:LLVM}, the first abstraction layer lays on the 
Clang AST. This representation allows to identify the regions of interest 
marked in the code. Pipeline follows by creating a three-address code or TAC 
representation, which facilitates the creation of a CDAG. It is here where 
patterns are searched, such as reductions. CDAG's operations and operands are 
packed, if possible, in order to generate vector operations. These are 
synthesized according to the SIMD backend if and only if 
cost model permits attending to its policy and the architectural features.

\section{MACVETH Expressions: MVExpr}
Clang implements complex expressions in order to handle any form or type of
code. They
provide many possibilities
when it comes to parse exactly the code. Nonetheless, MACVETH simplifies by 
wrapping all categories onto a small set of expressions. MVExpr is an abstract 
class that can be specialized for any type we
want to represent from the Clang AST, or to generate other abstractions. 
Besides, the idea of this class is to
provide a set of non-standard transformations for the expressions, e.g.
unrolling. Thus, MVExpr are instantiated using a factory.

We have implemented the following specializations, which are enough in order to
represent any value referenced in the regions of interest:

\begin{itemize}
	\item \textbf{MVExprArray}: represents any N-dimensional expression in the
	      code. It
	      holds information of the number of dimensions and name or value of the
	      indices. This class is very useful when performing unrolling, as it 
	      provides methods for computing deltas between two indices of the same 
	      or different form, e.g. for two positions in the array $A$ with 
	      affine indices $(i+1)*4$ and $(i+1)*2$, difference $(i+1)*4 - (i+1)*2 
	      > 1 \equiv$ are not
	      contiguous in memory.
	\item \textbf{MVExprVar}: regardless the type of the variable, this
	      abstraction
	      represents, basically, any \texttt{DeclRefExpr} from the Clang AST.
	\item \textbf{MVExprFunc}: this is a recursive abstraction which holds the
			name of the function and the parameters it receives. Parameters are 
			also MVExpr type. 

	\item \textbf{MVExprLiteral}: any number (integer, float, double), char, 
	etc. value that do not fit on any of the other abstractions.
\end{itemize}

\section{Three-Address Code IR: TAC}
Statements may differ in number of operations, data handled, or even types. 
This makes difficult to handle them properly.  The Three-Address Code (TAC) 
representation is used to translate any statement ($S$) into a
set of Single Stament Assignment (SSA), which have the same length. There are 
different ways of representing this format, but in our case we use quadruples, 
as described in Definition~\ref{def:TAC}.

\theoremstyle{definition}
\begin{definition}\label{def:TAC}
	A three-address code or TAC is a 4-tuple T=(a,b,c,$\oplus$) which 
	represents the assignment of $b \oplus c$ operation as $a=b \oplus c$. 
	If the $\oplus$ operator is unary, then $c$ is null, so $a=\oplus b$.
\end{definition}

Thus, any statement of a program is composed by a concatenation of operations, 
e.g. assignment ($=$, $+$, a function, etc.), 
which are 
split onto TACs respecting  their operational order. Thus, when any statement 
generates more than one TAC, temporal registers are generated, in a 
SSA-fashion. Those assignments are 
responsible for connecting TACs and, therefore, represent the logical order of 
the original statement. In essence, those TAC connections generate a tree 
structure. In order to 
perform this translation, we have 
implemented a recursive process which is listed in 
Algorithm~\ref{alg:stmtToTAC}. For the minimum case of an induction, i.e. the 
assignment $p = q$, the process remains, and result would be $T = (p,p,q,=)$. 
This property makes also possible to easily identify assignments, as it is only 
needed to test if $a=b$, i.e. the first two terms of the tuple.

\begin{algorithm}[h]\label{alg:stmtToTAC}
	\SetAlgoLined
	\KwIn{Stmt S}
	\KwResult{Set of TAC}
	L $\leftarrow$ \{\}\;
	Res $\leftarrow$ getResultOrTempReg(S)\;
	Lhs $\leftarrow$ getLHS(S)\;
	Rhs $\leftarrow$ getRHS(S)\;
	\If{isNonTerminal(Lhs)}{
		TAC $\leftarrow$ translateStmtToTAC(Lhs)\;
		addTacToList(L, TAC)\;
		Lhs $\leftarrow$ TAC.Res\;
	}
	\If{isNonTerminal(Rhs)}{
		TAC $\leftarrow$ translateStmtToTAC(Rhs)\;
		addTacToList(L, TAC)\;
		Rhs $\leftarrow$ TAC.Res\;
	}
	addTacToList(L, \{Res, Lhs, Rhs, getOp(S)\})\;
	return L\;
	\caption{translateStmtToTAC}
\end{algorithm}

\begin{corollary}\label{cor:TAC}
	Any statement $S$ can be represented as a set of interconnected TACs:
	$S = \{T\}/\forall T_{i} \in S \Rightarrow \exists T_{j} / a^{T_i} = 
	(b^{T_j} 
	| c^{T_j}) \lor a^{T_j} = (b^{T_i} 
	| c^{T_i}) $
\end{corollary}

This representation is widely used in compilers. Main advantages reside
in the simplicity of handling operations with the same number of operands or 
length. Besides, this format is very easy to handle in 
programmatically terms. MACVETH uses a quadruple 
format, but it could be used another in its place. As a matter of fact, 
GCC/GNU uses 
a different three-address code representation called 
GIMPLE~\cite{GCC:GIMPLE}. This IR format is more complex, since it has to 
handle all kind of codes, considering conditional jumps, gotos, different types 
of loops, external libraries, etc. It has its own syntax, operands and 
operations. Nonetheless, MACVETH's, as it has a reduced and concrete set of 
scenarios to perform, keeps this IR as simple as possible. Besides, MACVETH's 
is built using Clang/LLVM Tooling library, which already builds a 
tree structure properly parsed and interpreted.

Unrolling is also performed using this format, following the iterative process
listed in Algorithm~\ref{alg:TACunrolling}. This mechanism permits to generate 
a new list of unrolled TACs conserving their identity, i.e. the execution order 
in the 
program. MACVETH's approach is to create new tuples when unrolling. Even 
though this is costly in terms of memory, it enables the ability to handle  
them individually. By this, another type of packing can be sorted when grouping 
operations and operands, apart from packing together unrolled expressions.

\begin{algorithm}[h]\label{alg:TACunrolling}
	\SetAlgoLined
	\KwIn{TAC list $T$, Unrolling Factor $UF$, Loop nests $LN$}
	\KwResult{TAC list $T'$}
	$T'$ = \{\}\;
	\ForEach{LN} {
		\For{step++ $<$ UF} {
			\ForEach{TAC in T} {
				NewTac = \{\}\;
				\ForEach{Expr in TAC} {
					NewExpr $\leftarrow$ unrollExpr(step, LN, Expr)\;
					NewTac $\leftarrow$ placeExprInTac(NewExpr)\;
				}
				$T'$ $\leftarrow$ add(NewTac)\;
			}
		}
	}
	return $T'$\;
	\caption{Unrolling TAC list}
\end{algorithm}


\section{Computation Directed Acyclic Graph (CDAG)}
When it comes to schedule the different TACs in the ROI of our program, we need
a representation which can handle the dependencies between the statements and
some structures that store the information about the placement of them in the
execution. For this purpose we use a Computation Directed Acyclic Graph or
CDAG~\cite{bib:CDAGdefinition}. Informally, it is a forest that represents the
TACs as a set of nodes, where
each node can be a memory operation (load, store) or any other type of
operation (addition, multiplication, built-in function, etc.). Connections
between those nodes represent data dependencies. Formal
definition of the CDAG we have implemented can be found in
Definition~\ref{def:CDAG}. It is, essentially, a slight variation
of the definition given in~\cite{bib:CDAGdefinition}.

\theoremstyle{definition}
\begin{definition}\label{def:CDAG}
	A computation directed acyclic graph (CDAG) is a 4-tuple C=(I,V,E,O) of
	finite sets such that: (1) $I \subseteq V, O \subseteq (V-I)$; (2) $E
		\subseteq V
		\times V$ is the set of arcs; (3) $G=(V,E) \subseteq C$ is a subgraph 
		of C
\end{definition}

The importance of this structure is, in essence, to detect data races and to 
perform any kind of variation in the placement of operations, if possible. 
Nonetheless, this topic is not to be discussed in this work, since MACVETH uses 
the CDAG for sorting nodes in a topological order and detect patterns such as 
reductions. These patterns are found by traversing in reverse order the forest 
of operation nodes.

\begin{figure}[h]
	\centering
\begin{tikzpicture}[
	->,
	>=stealth',
	shorten >=.2pt,
	auto,
	node distance=2cm,
	thick,
	tmp node/.style={circle,draw,font=\large\bfseries},
	in node/.style={circle,draw=blue!100,font=\large\bfseries},
	out node/.style={circle,draw=red!90,font=\large\bfseries}
]

\node[in node] (1) {I};
\node[in node] (2) [right of=1] {I};
\node[tmp node] (3) [below of=2] {T};
\node[in node] (4) [right of=3] {I};
\node[tmp node] (5) [below of=4] {T};
\node[in node] (6) [right of=5] {I};
\node[tmp node] (7) [below of=6] {T};
\node[in node] (8) [right of=7] {I};
\node[out node] (9) [below of=8] {O};
%\node[main node] (1) {In};
%\node[main node] (2) [below left of=1] {2};
%\node[main node] (3) [below right of=2] {3};
%\node[main node] (4) [below right of=1] {4};

\path[every node/.style={font=\sffamily\small}]
(1) edge node [left] {} (3)
(2) edge node [left] {} (3)
(3) edge node [left] {} (5)
(4) edge node [left] {} (5)
(5) edge node [left] {} (7)
(6) edge node [left] {} (7)
(7) edge node [left] {} (9)
(8) edge node [left] {} (9)
;
%(1) edge node [left] {0.6} (4)
%%edge [bend right] node[left] {0.3} (2)
%edge [loop above] node {0.1} (1)
%(2) edge node [right] {0.4} (1)
%edge node {0.3} (4)
%edge [loop left] node {0.4} (2)
%edge [bend right] node[left] {0.1} (3)
%(3) edge node [right] {0.8} (2)
%edge [bend right] node[right] {0.2} (4)
%(4) edge node [left] {0.2} (3)
%edge [loop right] node {0.6} (4)
%edge [bend right] node[right] {0.2} (1);
\end{tikzpicture}
\caption{Graphical example of a reduction in a CDAG}
\label{fig:GraphCDAG}
\end{figure}

Figure~\ref{fig:GraphCDAG} shows the graphical representation of a CDAG 
generated from the code. Key idea of this IR is to provide information 
regarding the logical order of each statement, i.e. sort statements in order to 
pack them respecting data races, minimizing data movements and maximizing 
vector occupancy. Even though, not all optimizations regarding packaging are 
performed at this stage.

\section{VectorIR}
In order to tackle all different architectures when generating instructions,
we need a generic vector representation of the vector instructions we want to
pack and represent in our program. For this purpose, MACVETH uses the VectorIR, 
that basically
wraps a set of nodes from the CDAG onto a common structure which represent a 
vector operation. This vector representation is linked to the target 
architecture and ISA, as the maximum number of elements in a vector operation 
depends on the type of data and the maximum vector width in the architecture. 
Thus, each vector operation will also depend of two vector operands, which are, 
as well, a wrap of set of memory or result nodes in the CDAG. 
Definition~\ref{def:VectorIR} synthesis in a more formal manner this paragraph.

\theoremstyle{definition}
\begin{definition}\label{def:VectorIR}
	A vector operation is a 4-tuple $VIR=(\oplus, VOp_{R}, VOp_{1}, 
	VOp_{2})$ such that: (1) $|VOp_{i}| \leq ISA_{width}$, (2) $VOp_{i} 
	\subseteq C$, (3) $\forall i,j / VOp_{i} \cap VOp_{j} = \emptyset$
\end{definition}

To wrap up: VectorIR is a generic way of representing vector operations for the 
different
architectures. Because of this, at this stage there is no fusing operations or
any other kind of target-specific optimizations. The concrete backend will be in
charge of doing this. For instance, AVX introduced the fuse add-multiplication
instructions often called FMAs, which as their name suggest fuse additions and
multiplication onto a single operation; however they are only available for FP
operations and, therefore, not for integers. It would be pointless to tackle
all these issues in this representation that is why this is done in the AVX
backend instead.


