


MACVETH compiler handles different abstraction levels or intermediate
representations in order to facilitate the vectorization process. As MACVETH is 
based in Clang/LLVM~\cite{bib:clang}, the first abstraction layer lays on the 
Clang AST. This representation allows to identify the regions of interest for 
the translation, to extract 

\section{MACVETH Expressions: MVExpr}
Clang implements complex expressions in order to handle any form or type of
code. They
provide many possibilities
when it comes to parse exactly the code. Nonetheless, in our case we are
interested in a small set of operations, so we created basically a wrap for this
purpose. MVExpr is an abstract class that can be specialized for any type we
want to represent from the Clang AST. Besides, the idea of this class is to
provide a set of non-standard transformations for the expressions, e.g.
unrolling. Thus, MVExpr are instantiated using a factory.

We have implemented the following specializations, which are enough in order to
represent any value reference:

\begin{itemize}
	\item \textbf{MVExprArray}: represents any N-dimensional expression in the
	      code. It
	      holds information of the number of dimensions and name or value of the
	      indices. This class is very useful when performing unrolling, as it 
	      provides methods for computing deltas between two indices of the same 
	      or different form, e.g. $(i+1)*2$ vs. $(i+1)*4$.
	\item \textbf{MVExprVar}: regardless the type of the variable, this
	      abstraction
	      represents, basically, any \texttt{DeclRefExpr} from the Clang AST.
	\item \textbf{MVExprLiteral}: any integer, float, double, char, etc. value.
	\item \textbf{MVExprFunc}: this is a recursive abstraction which holds the
	      name of the function and the parameters it receives. Parameters are 
	      also MVExpr type.
\end{itemize}

\section{Three-Address Code IR: TAC}
Statements may differ in number of operations, data handled, or even types. 
This makes difficult to handle them properly.  The Three-Address Code (TAC) 
representation is used to translate any statement ($S$) into a
set of Single Stament Assignment (SSA), which have the same length. There are 
different ways of representing this format, but in our case we use quadruples, 
as described in Definition~\ref{def:TAC}.

\theoremstyle{definition}
\begin{definition}\label{def:TAC}
	A three-address code or TAC is a 4-tuple TAC=(a,b,c,$\oplus$) which 
	represents the 	assignment of the $b \oplus c$ operation as $a=b \oplus c$. 
	If the $\oplus$ operator is unary, then $c$ is null, so $a=\oplus b$.
\end{definition}

Thus, any statement of a program is composed by a concatenation of operations, 
e.g. assignment ($=$, $+$, a function, etc.), 
which are 
split onto TACs respecting  their operational order. Thus, when any statement 
generates more than one TAC, temporal registers are generated, in a 
SSA-fashion. Those assignments are 
responsible for connecting TACs and, therefore, represent the logical order of 
the original statement. In essence, those TAC connections generate a tree 
structure. In order to 
perform this translation, we have 
implemented a recursive process which is listed in 
Algorithm~\ref{alg:stmtToTAC}. 

\begin{algorithm}[h]\label{alg:stmtToTAC}
	\SetAlgoLined
	\KwIn{Stmt S}
	\KwResult{Set of TAC}
	L $\leftarrow$ \{\}\;
	Res $\leftarrow$ getResultOrTempReg(S)\;
	Lhs $\leftarrow$ getLHS(S)\;
	Rhs $\leftarrow$ getRHS(S)\;
	\If{isNonTerminal(Lhs)}{
		TAC $\leftarrow$ translateStmtToTAC(Lhs)\;
		addTacToList(L, TAC)\;
		Lhs $\leftarrow$ TAC.Res\;
	}
	\If{isNonTerminal(Rhs)}{
		TAC $\leftarrow$ translateStmtToTAC(Rhs)\;
		addTacToList(L, TAC)\;
		Rhs $\leftarrow$ TAC.Res\;
	}
	addTacToList(L, \{Res, Lhs, Rhs, getOp(S)\})\;
	return L\;
	\caption{translateStmtToTAC}
\end{algorithm}

\begin{corollary}\label{cor:TAC}
	Any statement $S$ can be represented as a set of interconnected TACs:
	$S = \{TAC\}$
\end{corollary}

This representation is widely used in compilers. The main advantages of it
resides in the simplicity of handling operations with the same number of
operands. Besides, this format is very easy to handle in programmatically terms.


Unrolling is also performed using this format, following the iterative process
listed in
Algorithm~\ref{alg:TACunrolling}.

\begin{algorithm}[H]\label{alg:TACunrolling}
	\SetAlgoLined
	\KwIn{TAC list $T$, Unrolling Factor $UF$, Loop nests $LN$}
	\KwResult{TAC list $T'$}
	$T'$ = \{\}\;
	\ForEach{LN} {
		\For{step = 0; step++ $<$ UF;} {
			\ForEach{TAC in T} {
				NewTac = \{\}\;
				\ForEach{Expr in TAC} {
					NewExpr $\leftarrow$ unrollExpr(step, LN, Expr)\;
					NewTac $\leftarrow$ placeExprInTac(NewExpr)\;
				}
				$T'$ $\leftarrow$ add(NewTac)\;
			}
		}
	}
	return $T'$\;
	\caption{Unrolling TAC list}
\end{algorithm}


\section{Computation Directed Acyclic Graph (CDAG)}
When it comes to schedule the different TACs in the ROI of our program, we need
a representation which can handle the dependencies between the statements and
some structures that store the information about the placement of them in the
execution. For this purpose we use a Computation Directed Acyclic Graph or
CDAG~\cite{bib:CDAGdefinition}. Informally, it is a forest that represents the
TACs as a set of nodes, where
each node can be a memory operation (load, store) or any other type of
operation (addition, multiplication, built-in function, etc.). Connections
between those nodes represent data dependencies. Formal
definition of the CDAG we have implemented can be found in
Definition~\ref{def:CDAG}. It is, essentially, a slight variation
of the definition given in~\cite{bib:CDAGdefinition}.

\theoremstyle{definition}
\begin{definition}\label{def:CDAG}
	A computation directed acyclic graph (CDAG) is a 4-tuple C=(I,V,E,O) of
	finite sets such that: (1) $I \subseteq V, O \subseteq (V-I)$; (2) $E
		\subseteq V
		\times V$ is the set of arcs; (3) $G=(V,E) \subseteq C$ is a subgraph 
		of C
\end{definition}

The importance of this structure is, in essence, to detect data races and to 
perform any kind of variation in the placement of operations, if possible. 
Nonetheless, this topic is not to be discussed in this work, since MACVETH uses 
the CDAG for sorting nodes in a topological order and detect patterns such as 
reductions.

\begin{figure}[h]
	\centering
\begin{tikzpicture}[
	->,
	>=stealth',
	shorten >=.2pt,
	auto,
	node distance=2cm,
	thick,
	tmp node/.style={circle,draw,font=\large\bfseries},
	in node/.style={circle,draw=blue!100,font=\large\bfseries},
	out node/.style={circle,draw=red!90,font=\large\bfseries}
]

\node[in node] (1) {I};
\node[in node] (2) [right of=1] {I};
\node[tmp node] (3) [below of=2] {T};
\node[in node] (4) [right of=3] {I};
\node[out node] (5) [below of=4] {O};
%\node[main node] (1) {In};
%\node[main node] (2) [below left of=1] {2};
%\node[main node] (3) [below right of=2] {3};
%\node[main node] (4) [below right of=1] {4};

\path[every node/.style={font=\sffamily\small}]
(1) edge node [left] {} (3)
(2) edge node [left] {} (3)
(3) edge node [left] {} (5)
(4) edge node [left] {} (5)
;
%(1) edge node [left] {0.6} (4)
%%edge [bend right] node[left] {0.3} (2)
%edge [loop above] node {0.1} (1)
%(2) edge node [right] {0.4} (1)
%edge node {0.3} (4)
%edge [loop left] node {0.4} (2)
%edge [bend right] node[left] {0.1} (3)
%(3) edge node [right] {0.8} (2)
%edge [bend right] node[right] {0.2} (4)
%(4) edge node [left] {0.2} (3)
%edge [loop right] node {0.6} (4)
%edge [bend right] node[right] {0.2} (1);
\end{tikzpicture}
\caption{Graphical example of CDAG}
\label{fig:GraphCDAG}
\end{figure}

Figure~\ref{fig:GraphCDAG} shows the graphical representation of a CDAG 
generated from the code. Key idea of this IR is to provide information 
regardin

\section{VectorIR}
In order to approach the different architectures when generating instructions,
we need a generic vector representation of the vector instructions we want to
have in our program. For this purpose, MACVETH uses the VectorIR, that basically
wraps a set of nodes from the CDAG onto a common structure which represent a 
vector operation.

To wrap up: VectorIR is a generic way of representing vector operations for the 
different
architectures. Because of this, at this stage there is no fusing operations or
any other kind of target-specific optimizations. The concrete backend will be in
charge of doing this. For instance, AVX introduced the fuse add-multiplication
instructions often called FMAs, which as their name suggest fuse additions and
multiplication onto a single operation; however they are only available for FP
operations and, therefore, not for integers. It would be pointless to tackle
all these issues in this representation that is why this is done in the AVX
backend instead.

\section{SIMDBackEnd: \textit{the} backend}
In order to synthesize the vector operations gathered in the VectorIR, there
must exists a target-specific backend for each ISA. MACVETH disposes a general
interface called \texttt{SIMDBackEnd}. This interface must be
implemented by each architecture and/or ISA in order to generate
target-specific code for each machine or system.

Furthermore, each backend has a table with costs for all instructions and 
operations, and their intrinsics equivalence, if they have. This table is 
generated from data gathered in~\cite{bib:uops} and~\cite{bib:intrinsics}. 
Thus, data for each ISA and architecture are provided to the compiler in 
separate files. It is at runtime when compiler decides what file to read.


