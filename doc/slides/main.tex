% This template has been adapted from LIM Lian Tze's version
%
% Universidade da Coruña. 2016
% Marcos Horro Varela

\documentclass[xcolor=table,hideothersubsections,aspectratio=1610]{beamer}
\usetheme{Gelugor}

% Packages
\usepackage[utf8]{inputenc}
%\usepackage[spanish]{babel}
\usepackage{helvet}
\usepackage[compatibility=false]{caption}
\usepackage{subcaption}
\usepackage{booktabs}
\usepackage{graphicx}
\usepackage{xcolor}
\usepackage{listings}
\usepackage{svg}
\setbeamercolor{block title alerted}{fg=white,bg=UDCpurple}
\usepackage{pdfpages}
\usepackage{fancyvrb}

%% Personal data
%\title[CSU Internship]{\color{white}\textbf{Platform-independent code synthesis techniques to generate platform-specific implementations}}
\title[CSU Internship]{\color{white}\textbf{Framework for Random Vector Packing}}

%\title[CSU Internship]{\color{white}\textbf{Characterizing codes in different architectures and developing a compiler for best-effort vectorization}}
%\subtitle{\color{white}Conceptos Básicos}
\subtitle{\color{white}}
\author[Marcos Horro]{
	Marcos Horro, Louis-Noël Pouchet\\
	University of A Coruña\\
	Colorado State University
} % Your name
\date{}

\begin{document}

% Cover frame
\begin{frame}[plain,t]
\titlepage
\end{frame}

\begin{frame}[plain]
\tableofcontents[hideallsubsections]
\end{frame}

% About me
%%%%%%%%%%%%%%%%%%%%%%%%%%%%%%%%%%%%%%
% %%%%%%%%%%%%%%%%%%%%%%%%%%%%%%%%%%%%%%
\begin{frame}[plain]
\textbf{Marcos Horro Varela}, PhD student at University of A Coruña in the Computer Architecture Group (GAC). My research lines are in the scope of high-performance computing (HPC), mainly: heterogeneous memory systems, hardware design exploration, computer architecture.

 \begin{itemize}
     \item gac.udc.es/\~{}marcos.horro
     \item marcos.horro@udc.gal
     \item github.com/markoshorro
     \item @markoshorro
 \end{itemize}
\end{frame}

% Introduction
%%%%%%%%%%%%%%%%%%%%%%%%%%%%%%%%%%%%%%
\section{Motivation}
%%%%%%%%%%%%%%%%%%%%%%%%%%%%%%%%%%%%%%
\frametitle{Motivation}
\begin{frame}{Motivation}
\begin{block}{Known issues}
\begin{itemize}
    \item Compilers' limitations regarding auto-vectorization for sparse codes: bad performance
    \item Need of characterize new architectures with non-disclosed information regarding SIMD performance (e.g. Microsoft Brainwave)
\end{itemize}

\end{block}
\begin{itemize}
    \item \textbf{Vectorization of irregular strides} is complex: compilers may not perform any action or have a detrimental impact $\rightarrow$ automation

        \item \textbf{Assessment of codes} may be tedious and complex: influence of dimensions (loop bounds, strides; data size, alignment) are not always clear $\rightarrow$ automation
        \item Different ISAs, different applications, different costs: in some cases we will want to vectorize in a certain way or just to not vectorize (\textbf{portability}) $\rightarrow$ automation
    %\item Implement a source-to-source compiler able to implement and use \textbf{trace scheduling techniques}, \textbf{architecture-independent} but \textbf{ISA-enriched}
\end{itemize}
\end{frame}

\begin{frame}{Objectives}
\begin{itemize}
    \item \textbf{Automate Benchmarking and Reverse-engineering with ML} $\rightarrow$ mini-framework written in Python for compiling, executing, measuring and analyzing (using WEKA) input codes and the dimensions of interest
    \item \textbf{Source-to-source SIMD compiler} $\rightarrow$ code synthesizer profile-based, using information collected with the framework commented before
\end{itemize}
\end{frame}



\begin{frame}[fragile]
\frametitle{Complexity of irregular codes}
\begin{columns}
\small
\begin{column}{0.45\textwidth}
\begin{verbatim}
for (int i = 0; i < N; ++i) {
    sum += x[i] * 42.3f
}
--------
for (int i = 0; i < N; i+=4) {
  vop0 = vset(42.3, ..., 42.3);
  vop1 = vload(&x[i]);
  vop2 = vmul(vop1, vop0);
  vop3 = vpermute(vop2, mask);
  vop4 = vadd(vop3, vop2);
  vop5 = vpermute(vop4, vop4,\ 
                        mask);
  vop6 = vadd(vop5, vop4);
  vstore(&sum, mask, vop6);
}




\end{verbatim}
\end{column}
\begin{column}{0.45\textwidth}  %%<--- here
\begin{verbatim}
for (int i = 0; i < N; ++i) {
    sum += x[4*i + 7] * 42.3f
}
--------
for (int i = 0; i < N; i+=4) {
  vop0 = vset(42.3, ..., 42.3);\end{verbatim}
{
\color{red}{
\vspace{-.6cm}
\begin{verbatim}  idx = (x*i+7, x*(i+1)+7,\
         x*(i+2)+7, x*(i+3)+7);
  // this is a wrap of 4 vload(op)
  vop1 = vgather(& x[i], idx);
\end{verbatim}
}}
\vspace{-.55cm}
\begin{verbatim}
  vop3 = vpermute(vop2, mask);
  vop4 = vadd(vop3, vop2);
  vop5 = vpermute(vop4, vop4,\ 
                        mask);
  vop6 = vadd(vop5, vop4);
  vstore(&sum, mask, vop6);
}


\end{verbatim}

\end{column}
\end{columns}
\end{frame}

\begin{frame}[fragile]
\frametitle{High-level algorithm}
\begin{verbatim}
INPUT: AST a
OUTPUT: AST w/SIMD intrinsics

  a_out <- clone(a)
  a_u   <- unrollAndJam(a, size)
  c     <- createCDAG(a_u)
  c     <- assignPlcmtToOps(c)
  {n}   <- groupNodesByPlace(c)
  forall node in {n} do
     cn <- packingNodesInVectorAndComputeCost(node, c)
     an <- SIMDSynthesis(cn)
     aout <- mergeReplace(au, an, aout)
  done
\end{verbatim}
    
\end{frame}


%% High level picture of the algorithm of the compiler

%% Unrolling
%% CDAG creation
%% Placement
%% Vector packing and cost computation or analysis
%% SIMD synthesis




% MACVETH compiler
%%%%%%%%%%%%%%%%%%%%%%%%%%%%%%%%%%%%%%
\section{MACVETH Compiler}
%%%%%%%%%%%%%%%%%%%%%%%%%%%%%%%%%%%%%%
\frametitle{MACVETH Compiler}
\begin{frame}{MACVETH Compiler [WIP]}
\begin{block}{Definition}
Multi-dimensional Array C-compiler for VEctorization and Translation in HPC applications: source-to-source compiler from C/C++ to SIMD intrinsics
\end{block}
\begin{itemize}
    \item Clang Tool (LLVM project), C/C++ ($\approx$ 3 KLOC)
    \item Pragma recognition (\texttt{\#pragma macveth}) to delimitate ROI
    \item Cost model computation: profile based
    \item Flexible instruction scheduling: interchangeable algorithms (Trace scheduling implementation (WIP))
    \item Different IR: TAC, CDAG, VectorIR, SIMD Generator $\rightarrow$ Vector API
    \item Abstract Vector API $\neq$ skeleton-based
\end{itemize}

\end{frame}

%%%%%%%%%%%%%%%%%%%%%%%%%%%%%%%%%%%%%%
\begin{frame}[fragile]
\frametitle{MACVETH: Vector IR}
\begin{figure}
    \centering
    \includegraphics[scale=.37]{img/CDAGVEC2.pdf}
    %\caption{CDAG from code}
\end{figure}
\end{frame}


%%%%%%%%%%%%%%%%%%%%%%%%%%%%%%%%%%%%%%
\subsection{Architecture}
\begin{frame}{MACVETH Architecture}
\begin{figure}
    \centering
    \includegraphics[scale=.775]{img/MACVETH.pdf}
\end{figure}
\end{frame}

%%%%%%%%%%%%%%%%%%%%%%%%%%%%%%%%%%%%%%
\begin{frame}{MACVETH: frontend}
\begin{block}{Clang Tool}
Clang Tools allows to create out-of-tree tools
\end{block}
\begin{itemize}
    \item Input code is parsed using the Clang AST
    \item Detection of region of interest by creating a subclass of \texttt{clang::PragmaHandler}
    \item Matching the loops and expressions within the region using Clang AST Matchers
\end{itemize}
\end{frame}


%%%%%%%%%%%%%%%%%%%%%%%%%%%%%%%%%%%%%%
\subsection{TAC}
\begin{frame}{MACVETH: TAC}
\begin{block}{Three-Address Code (TAC)}
This format is used for converting each statement in the region-of-interest to Single-Statement Assignment (SSA). It is represented as quadruples:
\begin{center}
    \texttt{a = b OP c}
\end{center}
\texttt{c} can be \texttt{nullptr} if \texttt{OP} is unary
\end{block}

\begin{itemize}
    \item Extended use in compilers
    \item Suitable for optimizations, e.g. fuse multiply-accumulate
    \item `Direct' translation onto graph (CDAG in our case)
\end{itemize}

\end{frame}

%%%%%%%%%%%%%%%%%%%%%%%%%%%%%%%%%%%%%%
\begin{frame}[fragile]
\frametitle{Three-Address Code examples}

\begin{verbatim}
a[i] = a[i] + b[i] * c[i]  ->    t1 = b[i] * c[i]
                               a[i] = a[i] + t1
                                
\end{verbatim}

\end{frame}

%%%%%%%%%%%%%%%%%%%%%%%%%%%%%%%%%%%%%%
\subsection{CDAG}
\begin{frame}{MACVETH: CDAG}
\begin{block}{Control Directered Acyclic Graph (CDAG)}
CDAG represents the dependencies between the statements (if any) by connecting nodes. CDAG handles two types of nodes: memory nodes and operational nodes.
\end{block}
\begin{itemize}
    \item \textbf{Memory nodes}: represent data that can be loaded (memory addresses) or set (literals, variables)
    \item \textbf{Operational nodes}: have at least one input node (up to two, any kind) and they represent an operation; they also hold a result value (that can also be a store to memory)
\end{itemize}
\end{frame}

%%%%%%%%%%%%%%%%%%%%%%%%%%%%%%%%%%%%%%
\begin{frame}[fragile]
\frametitle{MACVETH: CDAG}
\begin{verbatim}
a[i] = a[i] + b[i] * c[i]  ->    t1 = b[i] * c[i]
                               a[i] = a[i] + t1
                                
\end{verbatim}

\begin{figure}
    \centering
    \includegraphics[scale=.615]{img/CDAG.pdf}
\end{figure}
\end{frame}

%%%%%%%%%%%%%%%%%%%%%%%%%%%%%%%%%%%%%%
\begin{frame}{MACVETH: CDAG}
The potential utility of this representation is the ordering of the nodes:
\begin{itemize}
    \item Each node holds information regarding its \textbf{topological order} (or free scheduling)
    \item Trace scheduling (or any other algorithm/heuristic) can be used to \textbf{reorder the operations} in the program, e.g. potential improvements in vectorization by improving temporal and spatial locality (best use of vector operands)
    \item The CDAG is also responsible for computing the computational cost of the model
    \item Packing into vectors from different statements
    \item Implicitly parallel representation
\end{itemize}
\end{frame}

%%%%%%%%%%%%%%%%%%%%%%%%%%%%%%%%%%%%%%
\subsection{Vector IR}
\begin{frame}{MACVETH: Vector IR}
\begin{block}{Vector IR}
Intermediate representation which wraps operations and operands respecting a imposed order by the CDAG, e.g. free scheduling
\end{block}

\begin{itemize}
    \item Intermediate layer between the middle-end and the backend of the compiler to simplify the implementation of the Vector API
    \item Represents an architecture-agnostic set of vector operands and operations
    \item Hide the implementation of the CDAG nodes to the Vector API/SIMD generator
\end{itemize}

\end{frame}

%%%%%%%%%%%%%%%%%%%%%%%%%%%%%%%%%%%%%%
\begin{frame}[fragile]
\frametitle{MACVETH: Vector IR}
\begin{columns}
\begin{column}{0.5\textwidth}
\begin{verbatim}
Unrolled code:
a[i] = a[i] + b[i] * c[i]
a[i+1] = a[i+1] + b[i+1] * c[i+1]
a[i+2] = a[i+1] + b[i+2] * c[i+2]
a[i+3] = a[i+1] + b[i+3] * c[i+3]
\end{verbatim}
\end{column}
\begin{column}{0.35\textwidth}  %%<--- here
\begin{verbatim}

TAC: 
t1 = b[i] * c[i]
a[i] = a[i] + t1
t2 = b[i+1] * c[i+1]
a[i+1] = a[i+1] + t2
t3 = b[i+2] * c[i+2]
a[i+2] = a[i+2] + t3
t4 = b[i+3] * c[i+3]
a[i+3] = a[i+3] + t4
\end{verbatim}

\end{column}
\end{columns}
\end{frame}

%%%%%%%%%%%%%%%%%%%%%%%%%%%%%%%%%%%%%%
\begin{frame}[fragile]
\frametitle{MACVETH: CDAG and Vector IR}
\begin{figure}
    \centering
    \includegraphics[scale=.4]{img/FULLCDAG.pdf}
    %\caption{CDAG from code}
\end{figure}
\end{frame}

%%%%%%%%%%%%%%%%%%%%%%%%%%%%%%%%%%%%%%
\begin{frame}[fragile]
\frametitle{MACVETH: CDAG and Vector IR}
\begin{figure}
    \centering
    \includegraphics[scale=.37]{img/CDAGVEC1.pdf}
    %\caption{CDAG from code}
\end{figure}
\end{frame}

%%%%%%%%%%%%%%%%%%%%%%%%%%%%%%%%%%%%%%
\begin{frame}[fragile]
\frametitle{MACVETH: CDAG and Vector IR}
\begin{figure}
    \centering
    \includegraphics[scale=.37]{img/CDAGVEC2.pdf}
    %\caption{CDAG from code}
\end{figure}
\end{frame}


\subsection{CDAG Placement Algorithm}
\begin{frame}{CDAG Placement Algorithm}
\begin{block}{Key idea}
Iterative algorithm which computes the cost of a set of nodes in the CDAG \textbf{minimizing the total cost of the operations} and \textbf{maximizing the vector occupancy of the operands}
\end{block}

\begin{itemize}
    \item Naïve approach: ensure correctness of CDAG built $\rightarrow$ topological order
    \item Upcoming approaches: example-guided
\end{itemize}

\end{frame}


%%%%%%%%%%%%%%%%%%%%%%%%%%%%%%%%%%%%%%
\subsection{Vector API / SIMD Generator}
\begin{frame}{MACVETH: Vector API}
\begin{block}{Vector API}
Abstract interface for implementing vector operations by the backend, e.g.\textit{ vpack(data)}, \textit{vmul(op1, op2)}, \textit{vreduce(reduction, op)}
\end{block}
\begin{itemize}
    \item \textbf{Macro-based approach and ad-hoc approach}
    \item \textit{`Ease'} to implement complex operations as reductions more efficiently
    \item Trigonometric functions (such as inverse hyperbolic tangents) can be also synthesized as SIMD instructions
\end{itemize}
\end{frame}


% Characterizing affine codes
%%%%%%%%%%%%%%%%%%%%%%%%%%%%%%%%%%%%%%
\section{Characterizing affine codes}
%%%%%%%%%%%%%%%%%%%%%%%%%%%%%%%%%%%%%%
\frametitle{Characterizing affine codes}
\begin{frame}{Characterizing affine codes}
\begin{block}{Idea: $\mu$bench (ubench or microbench)}
Framework for the automation of compiling, executing programs and analyzing the performance results, with regard to a set of dimensions of interest.
\end{block}

\begin{itemize}
    \item Code not disclosed yet: Python 3, Makefile, YAML ($<$ 1 KLOC)
    \item Compile and execute a given code, which has a explicit region of interest (PolyBench style)
    \item A configuration file sets the dimensions of interest to be analyzed (e.g. ASM instructions, size of data, compiler flags, etc.)
    \item WEKA analyzes performance parameter with regard to the dimensions of interest by creating a decision tree (DT)
\end{itemize}

\end{frame}

%%%%%%%%%%%%%%%%%%%%%%%%%%%%%%%%%%%%%%
\subsection{Architecture}
\begin{frame}{High-level architecture}
\begin{figure}
    \centering
    \includegraphics[scale=.775]{img/ubencharch.pdf}
\end{figure}
\end{frame}

\subsection{microbench.py}
\begin{frame}{High-level architecture: microbench.py}
\begin{figure}
    \centering
    \includegraphics[scale=.6]{img/ubench.pdf}
\end{figure}
\end{frame}

%%%%%%%%%%%%%%%%%%%%%%%%%%%%%%%%%%%%%%
\begin{frame}[fragile]
\frametitle{microbench.py config file}
\begin{verbatim}
- general:
    input: code.c
    flags: -O3 -S ...
    sizes: 512 1024 2048 # -DN=<value>
    dimensions: I J K FLOPs Cycles
    init_val: 0 1 2 3 # -DINITVAL_<DIM>=<value>
    stride: 1 2 3 5 7 8 11 # -DSTRIDE_<DIM>=<value>
\end{verbatim}
\end{frame}


\subsection{weka.py}
\begin{frame}{High-level architecture: weka.py}
\begin{figure}
    \centering
    \includegraphics[scale=.6]{img/weka.pdf}
\end{figure}
\end{frame}

%%%%%%%%%%%%%%%%%%%%%%%%%%%%%%%%%%%%%%
\begin{frame}[fragile]
\frametitle{weka.py config file}
\begin{verbatim}
- general:
    input: test_data/data.csv
    ...
- prepare_data:
    cols: I J It Jt Is Js Jters Iters NELEMS
    rows: Vec_x=1
    pred: FLOPSs
    norm: ..
    cats: num: 10 ..
    dt_settings: ..
    min_acc: 0.8
- recommender:
    pred: FLOPSs
    interest_dim:
        value: 6.5 .. -> MAX_PERF
    dim: I J It Jt Is Js Jters Iters NELEMS
\end{verbatim}
\end{frame}

%%%%%%%%%%%%%%%%%%%%%%%%%%%%%%%%%%%%%%
\begin{frame}[fragile]
\frametitle{Example output}
\begin{verbatim}
[step 1] filtering csv... (test_data/data.csv)
[step 2] converting csv to arff format...
[step 3] creating training and testing sets... (nsets = 0)
[step 4] running experiments...
    [J48] train... 
    [J48] produce classfied outputs... 
    [J48] produce tester stats... 
[wrapper] displaying summary of results
summary of results: 
Correctly Classified Instances       16939         88.224  %
Mean absolute error                      0.0348
Relative absolute error                 19.6593 %
Size of the tree : 	301

\end{verbatim}
\end{frame}

\begin{frame}[fragile]
\frametitle{Example output}
\begin{verbatim}
Jters < 1.5 ...
|   |   |   |   Iters < 5
|   |   |   |   |   Iters < 3.5 : I-0.844-1.425 (10/0) [0/0]
|   |   |   |   |   Iters >= 3.5
|   |   |   |   |   |   Is < 2.5 : I-2.005-2.586 (7/0) [3/0]
|   |   |   |   |   |   Is >= 2.5
|   |   |   |   |   |   |   Is < 3.5 : I-0.844-1.425 (7/0) [3/0]
|   |   |   |   |   |   |   Is >= 3.5 : I-2.005-2.586 (7/0) [3/0]
Jters >= 1.5 ...
|   |   |   Iters < 7
|   |   |   |   Iters < 2.5 : I-3.747-4.328 (15/10) [5/2]
|   |   |   |   Iters >= 2.5 : I-4.328-4.908 (42/20) [8/3]
|   |   |   Iters >= 7
|   |   |   |   Iters < 10 : I-4.908-5.489 (7/0) [3/0]
|   |   |   |   Iters >= 10 : MAX_PERF (6/2) [4/0]
|   |   Jters >= 5 ...
|   |   |   Jters < 7
|   |   |   |   Is < 2.5 : I-4.908-5.489 (16/7) [4/1]
|   |   |   |   Is >= 2.5 : I-4.328-4.908 (14/5) [2/0]
|   |   |   Jters >= 7 : I-4.908-5.489 (57/30) [15/8]

\end{verbatim}
\end{frame}

\begin{frame}[fragile]
\frametitle{Example output}
\begin{verbatim}
[wrapper] executing recommender.py
cond 0: Is<=1.0 & Iters<=3.0 & I<=0.0 & Jters>4.0 & NELEMS>6.0 
cond 1: Jters>8.0 & It<=8.0 & It>4.0 & Iters<=4.0 & Iters>3.0 ...
cond 2: Jters>8.0 & Is<=4.0 & It>12.0 & Iters<=4.0 & Iters>3.0 ...
[recommender] false positives: 12/291 (4.12%)
	diff with cat I-4.804-5.419 is [2] (2)
    I  J It  Jt Is Js Jters Iters NELEMS
30  0  0  3  12  1  1    12     3     36
40  0  0  3  16  1  1    16     3     48
	diff with cat I-5.419-6.035 is [1] (10)
      I   J It  Jt Is Js Jters Iters NELEMS
20    0   0  3   8  1  1     8     3     24
...

\end{verbatim}
\end{frame}

\begin{frame}[fragile]
\frametitle{Example output}
\begin{verbatim}
[recommender] false negative: 147 
        I   J  It  Jt  Is  Js  Jters  Iters  NELEMS     FLOPSs
175     0   0   8   8   2   1      8      4      32  MAX_PERF
195     0   0   8  12   2   1     12      4      48  MAX_PERF
215     0   0   8  16   2   1     16      4      64  MAX_PERF
216     0   0   8  16   2   2      8      4      32  MAX_PERF
440     0   0  16   8   4   1      8      4      32  MAX_PERF
...    ..  ..  ..  ..  ..  ..    ...    ...     ...            ...
18111  16   7  16  16   4   2      8      4      32  MAX_PERF
18376  16  15   8  16   2   2      8      4      32  MAX_PERF
18600  16  15  16   8   4   1      8      4      32  MAX_PERF
18651  16  15  16  16   4   2      8      4      32  MAX_PERF
19140  16  16  16   8   4   1      8      4      32  MAX_PERF


\end{verbatim}
\end{frame}

%%%%%%%%%%%%%%%%%%%%%%%%%%%%%%%%%%%%%%


% Future work
%%%%%%%%%%%%%%%%%%%%%%%%%%%%%%%%%%%%%%
\section{Upcoming work}
%%%%%%%%%%%%%%%%%%%%%%%%%%%%%%%%%%%%%%
\frametitle{Upcoming work}
\begin{frame}{Upcoming work}
\begin{itemize}
    \item Outperform placement algorithms in the CDAG
    \item Development of new backends such as AVX512: improve portability
    \item Provide a clean interface between MACVETH and $\mu$bench
\end{itemize}
\end{frame}

% Cover frame
\begin{frame}[plain,t]
\titlepage
\end{frame}

\end{document}